\documentclass{beamer}
\usetheme{CambridgeUS}
\usepackage{graphicx,subcaption,lipsum}
% hide bottom navigation bar
\beamertemplatenavigationsymbolsempty
% Define names and their abbreviated versions
\newcommand\FirstName{Matej}
\newcommand\LastName{Kulháň}

%Information to be included in the title page:
\title{The Goodbooks Dtaset Clustering}
\author[\FirstName \space \LastName]
{\FirstName{} \LastName}

\begin{document}

\frame{\titlepage}

% \AtBeginSection[]
% {
%   \begin{frame}
%     \frametitle{Table of Contents}
%     \tableofcontents[currentsection]
%   \end{frame}
% }

\section{Data Preparation}
\begin{frame}{Dataset Overview}
    \begin{itemize}
        \item \textbf{Original Dataset}:
            \begin{itemize}
                \item \texttt{books.csv}
                \item \texttt{tags.csv}
                \item \texttt{book\_tags.csv}
            \end{itemize}
        \item \textbf{Extended Dataset}:
            \begin{itemize}
                \item \texttt{books\_enriched.csv}.
                \item Additional features:
                \begin{itemize}
                    \item Book descriptions
                    \item Pages
                \end{itemize}
            \end{itemize}
    \end{itemize}
\end{frame}


\begin{frame}{Data Preprocessing}
    \begin{itemize}
        \item \textbf{Genre Extraction}:
            \begin{itemize}
                \item Extracted the most frequent tags.
                \item Identified tags that corresponded to actual genres.
                \item Assigned a genre to books with the tag in their top 10 tags.
                \item Final list of genres:
                    \begin{itemize}
                        \item \texttt{young-adult}, \texttt{fantasy}, \texttt{nonfiction}, \texttt{romance},
                              \texttt{adult}, \texttt{science-fiction}, \texttt{contemporary},
                              \texttt{mystery}, \texttt{classics}, \texttt{historical-fiction}.
                    \end{itemize}
            \end{itemize}
        \item \textbf{Description Cleaning}:
            \begin{itemize}
                \item Cleaned \texttt{description} column for NLP tasks:
                    \begin{itemize}
                        \item Removed special characters, stopwords, etc.
                        \item Removed common entities like names, locations, etc.
                        \item Lemmatized the text.
                    \end{itemize}
                \item Filtered out non-English books.
            \end{itemize}
    \end{itemize}
\end{frame}

\section{Clustering}
\begin{frame}{Distance Matrix}
    \begin{itemize}
        \item \textbf{Handling Mixed Data Types}: 
            \begin{itemize}
                \item Used the \texttt{Gower} distance, which is specifically designed to handle datasets with a mix of numerical and categorical features.
                \item Allows for fair comparison across different feature types without needing to normalize all variables to the same scale.
            \end{itemize}
        \item \textbf{Features Used}:
            \begin{itemize}
                \item \textbf{Numerical Features}:
                    \begin{itemize}
                        \item \texttt{average\_rating}
                        \item \texttt{original\_publication\_year}
                        \item \texttt{pages}
                        \item \texttt{ratings\_count}
                        \item \texttt{genre\_count}
                    \end{itemize}
                \item \textbf{Binary Features}:
                    \begin{itemize}
                        \item \texttt{genres}
                    \end{itemize}
            \end{itemize}
    \end{itemize}
\end{frame}

\begin{frame}{Clustering}
    \begin{itemize}
        \item \textbf{K-Medoids Clustering}:
            \begin{itemize}
                \item Used the \texttt{K-Medoids} algorithm.
                \item Chose the number of clusters to be 12 based on a combination of the \texttt{elbow method} and human evaluation.
            \end{itemize}
    \end{itemize}
    \begin{figure}
        \centering
        \includegraphics[width=0.47\textwidth]{img/elbow_method_0.png}
        \caption{Elbow Method}
    \end{figure}
\end{frame}

\begin{frame}{Cluster Examples}
\begin{table}[]
\centering
\begin{tabular}{|c|c|c|p{6cm}|}
\hline
\textbf{Nonfic.} & \textbf{Fantasy} & \textbf{Pub. Year} & \textbf{Cluster Description} \\ \hline
0.01                & 0.07             & 1911                  & American literary classics about the pioneer lifestyle.     \\ \hline
0.00                & 0.81             & 1993                  & Science fiction about space exploration and intergalactic conflicts. \\ \hline
1.00                & 0.00             & 1985                  & Nonfiction books about leadership growth and personal development. \\ \hline
0.07                & 0.08             & 1850                  & Mystery books with a focus on crime investigation and characters. \\ \hline
\end{tabular}
\caption{Examples of clusters with selected columns and rounded data.}
\end{table}
\end{frame}

\section{Clustering With Text Embeddings}
    \begin{frame}{Text Embeddings}
        \begin{itemize}
            \item \textbf{Adding Semantic Context}:
                \begin{itemize}
                    \item Used text embeddings to capture the nuanced meaning of book descriptions.
                    \item Helped differentiate books within broad genres (e.g., various types of nonfiction).
                \end{itemize}
            \item \textbf{SBERT Embeddings}:
                \begin{itemize}
                    \item Used the \texttt{Sentence-BERT} model to generate embeddings.
                    \item Calculated the distance matrix using the \texttt{cosine} distance.
                    \item Combined the Gower distance matrix (from numerical and binary features) with the distance matrix derived from text embeddings using a weighted sum approach.
            \end{itemize}
    \end{itemize}
\end{frame}

\begin{frame}{Clustering}
    \begin{itemize}
        \item \textbf{K-Medoids Clustering}:
            \begin{itemize}
                \item Used the \texttt{K-Medoids} algorithm.
                \item Chose the number of cluster to be \texttt{14} based on a combination of the \texttt{elbow method} and human evaluation.
            \end{itemize}
    \end{itemize}
    \begin{figure}
        \centering
        \includegraphics[width=0.47\textwidth]{img/elbow_method_0.5.png}
        \caption{Elbow Method}
    \end{figure}
\end{frame}

\begin{frame}{Cluster Examples}
\begin{table}[]
\centering
\begin{tabular}{|c|c|c|p{6cm}|}
\hline
\textbf{Nonfic.} & \textbf{Fantasy} & \textbf{Pub. Year} & \textbf{Cluster Description} \\ \hline
0.92                & 0.03             & 1972                  & Inspirational nonfiction about personal faith and spiritual growth.     \\ \hline
0.00                & 0.71             & 1992                  & Science fiction about space exploration and technological anomalies. \\ \hline
0.96                & 0.00             & 1977                  & Intellectual nonfiction on leadership and personal development. \\ \hline
0.03                & 0.05             & 1985                  & Whimsical children's stories with humorous and relatable protagonists. \\ \hline
\end{tabular}
\caption{Examples of clusters with selected columns and rounded data.}
\end{table}
\end{frame}

\section{Naming the Clusters}
\begin{frame}{Cluster Naming Process}
    \begin{itemize}
        \item \textbf{Extracting Keywords:}
            \begin{itemize}
                \item Used the \texttt{Tf-Idf} algorithm to identify the most relevant keywords from book descriptions.
            \end{itemize}
        \item \textbf{Generating Cluster Names:}
            \begin{itemize}
                \item Used a pre-trained \texttt{Large Language Model (LLM)} for generating descriptive and human-readable cluster names.
                \item The model was instructed to focus on conciseness and relevance based on extracted keywords.
            \end{itemize}
        \item \textbf{Example}:
            \begin{itemize}
                \item Keywords: space, galaxy, aliens, technology
                \item Generated Name: \texttt{Science fiction about space exploration.}
            \end{itemize}
    \end{itemize}
\end{frame}

\section{Future Work}
\begin{frame}{Future Work}
    \begin{itemize}
        \item \textbf{Clustering}:
            \begin{itemize}
                \item Explore different clustering algorithms such as \texttt{HDBSCAN} and compare their performance.
            \end{itemize}
        \item \textbf{Evaluation}:
            \begin{itemize}
                \item Move beyond visual inspection by implementing systematic methods to evaluate cluster quality in a reproducible and scalable manner.
            \end{itemize}
        \item \textbf{Keyword Extraction}:
            \begin{itemize}
                \item Experiment with improved approaches for extracting meaningful keywords from book descriptions, such as:
                    \begin{itemize}
                        \item Rule-based techniques like \texttt{RAKE}.
                        \item Leveraging the power of \texttt{Pre-trained LLMs}.
                    \end{itemize}
            \end{itemize}
    \end{itemize}
\end{frame}

\end{document}

